\section{Zárószó}

\begin{frame}
    \frametitle{Zárószó}

    A kód megtalálható a \href{https://gitlab.com/rockdonald2/community-space}{gitlab.com/community-space} oldalon.

    \medbreak

    Ahol a kód 3 részben van szétbontva:
    \begin{itemize}
        \small
        \item \texttt{backend/} mappa tartalmazza a projekt micro-service-einek kódját, jelenleg 5 (+ 1) ilyen projekt van ezen belül (5 service + 1 library, mind \texttt{Java Spring}-ben);
        \item \texttt{frontend/} mappa tartalmazza a \texttt{NextJS} front-end kódot;
        \item \texttt{devops/} mappa tartalmazza a \texttt{Terraform}, \texttt{Helm}, és \texttt{CI/CD} kódrészleteket, amelyek megvalósítják a folyamatos fejlesztést és kitelepítést (minden commit után automatikus build és Docker deploy folyamatok a projekt \texttt{Gitlab Registry}-jére), infrastructure-as-code a K8S cluster létrehozásához és \texttt{Helm Chart} a könnyű K8S kitelepítésért.
    \end{itemize}

    \medbreak

    What's missing? És mit lehetne még hozzáadni a jövőben:
    \begin{itemize}
        \small
        \item Hub-ok manipulációja (kliens oldalon csak létrehozni lehet, külön oldal);
        \item Konfigurálható emlékeztetők memo-kra;
        \item Chat funkcionalitás?
        \item Adminisztrátor szerepkörök (a hub owner-ek fölött álló szerepkör);
        \item \dots
    \end{itemize}

\end{frame}