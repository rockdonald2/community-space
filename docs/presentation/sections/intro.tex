\section{CS Bevezető}

\begin{frame}
    \frametitle{Bevezető és CS projekt hatóköre}

    Tudásmegosztásra kihelyezett közösségi munkaplatform a vállalati/egyéb közösségen belüli gyors információmegosztásra és megőrzésre rövid memo-k formájában, ahol bárki könnyedén megoszthatja aktuális gondolatait/felhívásait/közléseit egy adott tematikáról vagy gondolatról csoportokon belül, amelyeket mások teljesíthetnek és elvégzésüket kvázi kötelezővé tehetik határidők megadásával, amelyekre a felhasználók értesítést kapnak sok más egyéb mellett.

    \medbreak

    \begin{block}{Megjegyzés}
        Hub több memo-t foglal magába és tagok csoportjaként fogható fel, a memo egy általában rövid üzenet, amely hasonlít az e-mail-hez, gyakran valamilyen elvégzendő feladatot fog meghatározni.
    \end{block}
\end{frame}

\begin{frame}
    \frametitle{Bevezető és CS projekt hatóköre}

    Alapvető funkcionalitások és üzleti folyamatok:
    \begin{itemize}
        \item Felhasználók képesek contókat létrehozni a platformhoz való csatlakozáshoz;
        \item Saját közösségeket, hub-okat létrehozni, meglévőkhöz csatlakozási kérelmeket leadni, amelyeket adminisztrátor felhasználók fogadhatnak/utasíthatnak el;
        \item Felhasználók képesek memo-kat megosztani különböző láthatósági szintekkel kizárólag a hub-on belül, ezek elolvasni, módosítani, archívumba helyezni, törölni, kitűzni, teljesíteni, keresni közöttük, stb.;
        \item Aktivitási hőtérkép és interakciós (milyen tevékenységek végződtek el a múltban) listázás az aktuális hétre vetítve;
        \item Felhasználók képesek összegzés formájában megtekinteni az összes hub-jukat, informálódni arról, hogy hány új memo van, hány aktív tag, stb.;
        \item \textbf{Real-time} értesítési mechanizmus, platformbéli aktív felhasználók kilistázása.
    \end{itemize}

    \begin{block}{Megjegyzés}
        Egyéb kiegészítő (technikai) funkcionalitások az automatizált \texttt{GitLab CI/CD}, a felhő \texttt{K8S}, annak menedzsmentje \texttt{Terraform}-al, és a \texttt{Helm Chart} kitelepítés.
    \end{block}
\end{frame}